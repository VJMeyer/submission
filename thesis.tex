% Main template for University of Bayreuth thesis
\documentclass[12pt,a4paper]{report}      %   vs book

% Import custom packages and settings


% Set page geometry
\usepackage[a4paper,margin=2.5cm]{geometry}
\usepackage{graphicx}
\usepackage{booktabs}
\usepackage{bookmark}
\usepackage{hyperref}

% Define custom commands for metadata
\newcommand{\subtitle}{An Epistemic Framework for Leveraging Frontier AI
Systems to Upscale Conditional Policy Assessments in Bayesian Networks
on a Narrow Path towards Existential Safety}
\newcommand{\supervisorname}{Supervisor Name}
\newcommand{\fieldofstudy}{Philosophy \& Economics M.A.}
\newcommand{\matriculationnumber}{1828610}
\newcommand{\submissiondate}{May 26, 2025}
\newcommand{\wordcount}{30000}
\providecommand{\tightlist}{%
  \setlength{\itemsep}{0pt}\setlength{\parskip}{0pt}}

% Standard document begins
\begin{document}

% Create custom title page
\begin{titlepage}
\thispagestyle{empty}% Remove page number from title page

% Top header with logo (left) and department (right)
\begin{minipage}{0.3\textwidth}
  \includegraphics[width=5cm]{latex/uni-bayreuth-logo.png}
\end{minipage}
\hfill
\begin{minipage}{0.9\textwidth}
  \begin{center}
    -- P\&E Master's Programme --\\
    Chair of Philosophy, Computer\\
    Science \& Artificial Intelligence
  \end{center}
\end{minipage}

% Horizontal rule
\vspace{1.5cm}
\hrule
\vspace{2cm}

% Title in bold
\begin{center}
  \Large\textbf{Automating the Modelling of
Transformative Artificial Intelligence Risks}
\end{center}
\vspace{0.2cm}

\begin{center}
  -----
\end{center}
\vspace{0.2cm}

% Subtitle in italics with quotation marks
\begin{center}
  \normalsize``\textit{An Epistemic Framework for Leveraging Frontier AI Systems
to Upscale Conditional Policy Assessments in Bayesian Networks on a Narrow Path towards Existencial Safety }''
\end{center}
\vspace{0.2cm}

\begin{center}
  -----
\end{center}
\vspace{0.2cm}

% Thesis designation
\begin{center}
  A thesis submitted at the Department of Philosophy\\[0.4cm]
  for the degree of \textit{Master of Arts in Philosophy \& Economics}
\end{center}

\vspace{1.5cm}
% Horizontal rule
\hrule
\vspace{1.5cm}

% Author and supervisor information with precise alignment
\begin{minipage}[t]{0.48\textwidth}
  \textbf{Author:}\\[0.3cm]
  Valentin Jakob Meyer\\
  Valentin.meyer@uni-bayreuth.de\\
  \textit{Matriculation Number:} 1828610\\
  \textit{Tel.:} +49 (1573) 4512494\\
  Pielmühler Straße 15\\
  52066 Lappersdorf
\end{minipage}
\hfill
\begin{minipage}[t]{0.48\textwidth}
  \begin{flushright}
    \textbf{Supervisor:}\\[0.3cm]
    Dr. Timo Speith\\[0.3cm]
    \textit{Word Count:}\\
    30.000\\[0.15cm]
    \textit{Source / Identifier:}\\
    Document URL
  \end{flushright}
\end{minipage}

% Date at bottom
\vfill
\begin{center}
  26th of May 2025
\end{center}
\end{titlepage}

% \pagenumbering{arabic} % Switch to Arabic page numbering
% \setcounter{page}{1} % Reset page numbers to 1

% Rest of document
\chapter*{Abstract}\label{abstract}
\addcontentsline{toc}{chapter}{Abstract}

The coordination crisis in AI governance presents a paradoxical
challenge: unprecedented investment in AI safety coexists alongside
fundamental coordination failures across technical, policy, and ethical
domains. These divisions systematically increase existential risk by
creating safety gaps, misallocating resources, and fostering
inconsistent approaches to interdependent problems. This thesis
introduces AMTAIR (Automating Transformative AI Risk Modeling), a
computational approach that addresses this coordination failure by
automating the extraction of probabilistic world models from AI safety
literature using frontier language models.

The AMTAIR system implements an end-to-end pipeline that transforms
unstructured text into interactive Bayesian networks through a novel
two-stage extraction process: first capturing argument structure in
ArgDown format, then enhancing it with probability information in
BayesDown. This approach bridges communication gaps between stakeholders
by making implicit models explicit, enabling comparison across different
worldviews, providing a common language for discussing probabilistic
relationships, and supporting policy evaluation across diverse
scenarios.

\textsubscript{Source:
\href{https://VJMeyer.github.io/submission/thesis.qmd.html}{Article
Notebook}}

\section{Grading}\label{grading}

\subsection{Research (10\%)}\label{research-10}

\begin{itemize}
\tightlist
\item
  demonstrates knowledge of the subject area as drawn from appropriate
  sources
\item
  incorporates insights from in-class discussions
\item
  draws on appropriate additional materials beyond those covered in
  class (primary as well as secondary sources)
\item
  covers relevant material at appropriate level of detail
\end{itemize}

\section{Structure (10\%)}\label{structure-10}

\begin{itemize}
\item
  outlines project clearly in the introduction
\item
  discussion follows a logical order
\item
  order of sections corresponds to outline
\item
  uses appropriate topic and transition sentences
\item
  employs proper signposting and cross-referencing throughout paper
\item
  sections are appropriately numbered and headlined
\end{itemize}

\section{Callout Test --- Language \& Style
(10\%)}\label{callout-test-language-style-10}

\begin{itemize}
\item
  employs appropriate tone and academic language
\item
  uses effective and sophisticated sentence variety, diction, and
  vocabulary
\item
  employs correct English spelling and grammar
\item
  is clearly written and uses appropriate sentence complexity
\item
  communicates main points effectively / is easy to follow
\item
  formats citations and references correctly and consistently
  (e.g.~(AUTHOR, YEAR) citation style)
\item
  names all primary and secondary sources
\item
  includes a complete list of references with full bibliographic details
\end{itemize}

More text

\chapter{Introduction}\label{introduction}

\section{Introduction}\label{introduction-1}

10\% of Grade:

\begin{itemize}
\tightlist
\item
  introduces and motivates the core question or problem
\item
  provides context for discussion (places issue within a larger debate
  or sphere of relevance)
\item
  states precise thesis or position the author will argue for
\item
  provides roadmap indicating structure and key content points of the
  essay
\end{itemize}

\textasciitilde{} 14\% of text \textasciitilde{} 4200 words

\begin{itemize}
\tightlist
\item
  introduces and motivates the core question or problem
\end{itemize}

\section{Motivation: Problem
Statement}\label{motivation-problem-statement}

\section{Motivation: Research
Question}\label{motivation-research-question}

\begin{itemize}
\tightlist
\item
  provides context for discussion (places issue within a larger debate
  or sphere of relevance)
\end{itemize}

\section{Scope: Aim \& Context of the
Research}\label{scope-aim-context-of-the-research}

\section{Significance of the Research: Theory of
Change}\label{significance-of-the-research-theory-of-change}

• states precise thesis or position the author will argue for

\section{Thesis Statement \& Position: (Aim of the
Paper)}\label{thesis-statement-position-aim-of-the-paper}

\begin{itemize}
\tightlist
\item
  provides roadmap indicating structure and key content points of the
  essay
\end{itemize}

\section{Overview: Structure \& Approach of the Paper (Roadmap ---
Theory of
Change)}\label{overview-structure-approach-of-the-paper-roadmap-theory-of-change}

\section{Table of Contents}\label{table-of-contents}

\section{Problem Statement ---
Motivation}\label{problem-statement-motivation}

Continued AI Progress:

\begin{itemize}
\tightlist
\item
  Rapid advancements in AI technology increase both potential benefits
  and risks.
\end{itemize}

Existential Risks (AI X-Risk):

\begin{itemize}
\tightlist
\item
  Advanced AI systems could pose significant threats if misaligned with
  human values.
\end{itemize}

Complexity Challenges:

\begin{itemize}
\tightlist
\item
  The intricate nature of AI systems complicates policy formulation and
  understanding.
\end{itemize}

Limitations of Current Approaches:

\begin{itemize}
\tightlist
\item
  MTAIR's Reliance on Human Labor:

  \begin{itemize}
  \tightlist
  \item
    Modeling Transformative AI Risks (MTAIR) is constrained by manual
    cognitive efforts.\\
  \end{itemize}
\item
  Need for Automation:

  \begin{itemize}
  \tightlist
  \item
    Scaling and automating risk modeling is essential to keep pace with
    AI developments.
  \end{itemize}
\end{itemize}

Opportunity:

\begin{itemize}
\tightlist
\item
  Leveraging new technologies to enhance our ability to model and
  mitigate AI risks.
\end{itemize}

\section{Aim of the Paper}\label{aim-of-the-paper}

\subsection{Research Question \& Scope}\label{research-question-scope}

\subsubsection{Can frontier AI technologies be utilized to automate the
modeling of transformative AI risks, so as to allow for the prediction
of policy
impacts?}\label{can-frontier-ai-technologies-be-utilized-to-automate-the-modeling-of-transformative-ai-risks-so-as-to-allow-for-the-prediction-of-policy-impacts}

Frontier AI Technology: Today's most capable AI systems (e.g.~GPT4 level
LLMs)\\
Scaling Up: Automating the previously ``manual'' cognitive labor\\
Modeling: Formalizing the world views underlying arguments\\
Transformative AI: Level of AI capabilities defined by severe impact on
the world\\
Safety \& Governance Literature: Publications, reports etc. concerned
with risks from AI

Automated Estimation: Non-manual (AI systems + scaffolding), quantified
evaluations\\
Probability Distributions: Formal expressions of the expectations over
future worlds\\
Conditional Trees of Possible Worlds: ``If \ldots{} then\ldots{}''
reasoning over ways things may play out\\
Forecasting Policy Impacts: Qualitative \& quantitative evaluation of
expected outcomes

\subsection{Significance of the
Research}\label{significance-of-the-research}

\subsection{}\label{section}

\section{Theory of Change --- Approach \& Structure of the
Paper}\label{theory-of-change-approach-structure-of-the-paper}

Multiplicative Benefits:

\begin{itemize}
\tightlist
\item
  Automation × Live Prediction Market Integrations × Policy Impact
  Evaluations
\end{itemize}

Explanation:\\
Automation:

\begin{itemize}
\item
  Increases efficiency and scalability of risk modeling.

  Live Prediction Markets:
\item
  Provides up-to-date, collective intelligence to inform models.

  Policy Impact Evaluations:
\item
  Improves the accuracy and relevance of policy assessments.
\end{itemize}

Outcome:

\begin{itemize}
\tightlist
\item
  Enhanced ability to develop effective policies that mitigate AI risks.
\end{itemize}

Visual Aid:

\begin{itemize}
\tightlist
\item
  A diagram illustrating how each component amplifies the others,
  leading to greater overall impact.
\end{itemize}

\section{}\label{section-1}

\section{Overview / Table of Contents}\label{overview-table-of-contents}

\textsubscript{Source:
\href{https://VJMeyer.github.io/submission/chapters/Introduction.qmd.html\#f19d721c-c76a-4263-9263-58fc63674fd2}{Introduction}}

\chapter{Context}\label{context}

\subsection{20\% of Grade:}\label{of-grade}

\begin{itemize}
\item
  demonstrates understanding of all relevant core concepts
\item
  explains why the question/thesis/problem is relevant in student's own
  words (supported by quotations)
\item
  situates it within the debate/course material
\item
  reconstructs selected arguments and identifies relevant assumptions
\item
  describes additional relevant material that has been consulted and
  integrates it with the course material as well as the research
  question/thesis/problem
\end{itemize}

\textasciitilde{} 29\% of text \textasciitilde{} 8700 words

\begin{enumerate}
\def\labelenumi{\arabic{enumi}.}
\tightlist
\item
  successively (chunk my chunk) introduce concepts/ideas --- and 2.
  ground each with existing literature
\end{enumerate}

\section{Theoretical Background
Considerations}\label{theoretical-background-considerations}

\subsection{DAG / BayesNets}\label{dag-bayesnets}

\subsection{State of the art (MTAIR) ---
Explanation}\label{state-of-the-art-mtair-explanation}

\subsubsection{Carlsmith Model
(Analytica)}\label{carlsmith-model-analytica}

\subsection{(Intro) Example ---
Rain/Sprinkler/Lawn}\label{intro-example-rainsprinklerlawn}

/ Rain/Sprinkler/Lawn DAG / BayesNet --- Extended Example

\begin{verbatim}
…
\end{verbatim}

Own Position/Argument: AMTAIR \ldots{} Own Rain/Sprinkler/Lawn DAG /
BayesNet Implementation

\section{Methodology}\label{methodology}

MTAIR / Carlsmith Model (Analytica) --- Explanation (--- is motivation:
should come first)

\subsection{Kialo}\label{kialo}

\subsection{Rain/Sprinkler/Lawn DAG}\label{rainsprinklerlawn-dag}

\subsection{BayeServer}\label{bayeserver}

\subsection{BayesNet --- Extended
Example}\label{bayesnet-extended-example}

\subsection{Code + documentation}\label{code-documentation}

\textsubscript{Source:
\href{https://VJMeyer.github.io/submission/chapters/Context.qmd.html\#c72e7a53-afa2-42e6-8531-cb6abbeedfdb}{Context}}

\chapter{AMTAIR}\label{amtair}

\subsection{20\% of Grade: \textasciitilde{} 29\% of text
\textasciitilde{} 8700 words}\label{of-grade-29-of-text-8700-words}

\begin{itemize}
\tightlist
\item
  provides critical or constructive evaluation of positions introduced
\item
  develops strong (plausible) argument in support of author's own
  position/thesis
\item
  argument draws on relevant course material claim/argument
\item
  demonstrate understanding of the course materials incl.~key arguments
  and core concepts within the debate
\item
  claim/argument is original or insightful, possibly even presents an
  original contribution to the debate
\end{itemize}

\section{Own Carlsmith Model Implementation ---
Explanation}\label{own-carlsmith-model-implementation-explanation}

\section{Own Implementation: Good example from a published
paper}\label{own-implementation-good-example-from-a-published-paper}

\section{Implementation}\label{implementation}

TestText

\section{Results}\label{results}

TestText

\textsubscript{Source:
\href{https://VJMeyer.github.io/submission/chapters/AMTAIR.qmd.html\#b6d1ccc7-d5e4-4c44-a778-672548c30c19}{AMTAIR}}

\chapter{Discussion}\label{discussion}

\section{Discussion}\label{discussion-1}

10\% of Grade: \textasciitilde{} 14\% of text \textasciitilde{} 4200
words

\begin{itemize}
\tightlist
\item
  discusses a specific objection to student's own argument
\item
  provides a convincing reply that bolsters or refines the main argument
\item
  relates to or extends beyond materials/arguments covered in class
\end{itemize}

\chapter{Discussion --- Exchange, Controversy \&
Influence}\label{discussion-exchange-controversy-influence}

\section{Challenges \& Problems --- Red Teaming Problems, Failures \&
Downsides}\label{challenges-problems-red-teaming-problems-failures-downsides}

\begin{verbatim}
Potential Failures:
\end{verbatim}

\begin{itemize}
\item
  Data Issues: Inaccurate or biased inputs.
\item
  Model Limitations: Oversimplifications.
\item
  Tech Risks: AI misinterpretations.

  Red Teaming:
\item
  Stress-testing models to find weaknesses.

  Impact on Theory of Change:
\item
  Identifying points of failure strengthens the approach.
\end{itemize}

\section{Implications \& Impact --- Uptake, Feedback Loops, Uptake \&
Success -- Green Teaming
--}\label{implications-impact-uptake-feedback-loops-uptake-success-green-teaming}

\begin{verbatim}
Potential Outcomes:
\end{verbatim}

\begin{itemize}
\item
  First-Order: Reduced AI risks through better policies.
\item
  Second-Order: Enhanced collaboration.
\item
  Third-Order: Framework applied to other global risks.

  Feedback Loops:
\item
  Continuous model improvement.
\item
  Adaptive policy-making.

  Green Teaming:
\item
  Strategies to maximize positive impacts.
\end{itemize}

\section{Known Unknowns \& Unknown Unknowns --- Input Data Example:
Modeling Author Worldviews from Bibliographies Instead of Individual
Papers}\label{known-unknowns-unknown-unknowns-input-data-example-modeling-author-worldviews-from-bibliographies-instead-of-individual-papers}

\begin{verbatim}
Potential Outcomes:
\end{verbatim}

\begin{itemize}
\item
  First-Order: Reduced AI risks through better policies.
\item
  Second-Order: Enhanced collaboration.
\item
  Third-Order: Framework applied to other global risks.

  Feedback Loops:
\item
  Continuous model improvement.
\item
  Adaptive policy-making.

  Green Teaming:
\item
  Strategies to maximize positive impacts.
\end{itemize}

\textsubscript{Source:
\href{https://VJMeyer.github.io/submission/chapters/Discussion.qmd.html\#24e67ea3-9f1b-4065-99fd-760f4f1a1313}{Discussion}}

\chapter{Conclusion}\label{conclusion}

\section{The Current State of Things \& How to
Continue}\label{the-current-state-of-things-how-to-continue}

10\% of Grade: \textasciitilde{} 14\% of text \textasciitilde{} 4200
words

\begin{itemize}
\tightlist
\item
  summarizes thesis and line of argument
\item
  outlines possible implications
\item
  notes outstanding issues / limitations of discussion
\item
  points to avenues for further research
\item
  overall conclusion is in line with introduction
\end{itemize}

\section{Summary --- Key Takeaways \&
Findings}\label{summary-key-takeaways-findings}

\subsection{Assessing Policy Effects:}\label{assessing-policy-effects}

Evaluating how different policies alter P(Doom).

\subsection{Conditional Probability:}\label{conditional-probability}

Calculating P(Doom \textbar{} Policy Alpha).

\subsection{Methodology:}\label{methodology-1}

Update model parameters based on policy implementation.

Recompute probabilities accordingly.

\subsection{Purpose:}\label{purpose}

Inform policymakers of potential policy effectiveness.

Prioritize interventions that significantly reduce risks.

\section{Outlook --- Outlook \& Next Steps / Further
Research}\label{outlook-outlook-next-steps-further-research}

\subsection{Scaling Up:}\label{scaling-up}

\begin{itemize}
\tightlist
\item
  Include more variables and data sources.
\end{itemize}

\subsection{Collaboration:}\label{collaboration}

\begin{itemize}
\tightlist
\item
  Partner with policymakers and researchers.
\end{itemize}

\subsection{Technological
Enhancements:}\label{technological-enhancements}

\begin{itemize}
\tightlist
\item
  Employ advanced AI techniques.
\end{itemize}

\subsection{Potential Impact:}\label{potential-impact}

\begin{itemize}
\item
  Influence global AI governance.

  \subsection{Limitations of the
  Analysis}\label{limitations-of-the-analysis}
\end{itemize}

\subsection{Policy Implications \&
Recommendations}\label{policy-implications-recommendations}

\subsection{Areas for Future Research}\label{areas-for-future-research}

\subsection{Open Questions --- Central/Remaining Questions \&
Feedback}\label{open-questions-centralremaining-questions-feedback}

\subsubsection{Questions:}\label{questions}

\begin{itemize}
\tightlist
\item
  How can we improve automation accuracy?\\
\item
  What challenges exist in policy implementation?\\
\item
  How do we mitigate AI model biases?\\
\item
  How can interdisciplinary efforts enhance outcomes?
\end{itemize}

\subsubsection{Feedback:}\label{feedback}

\begin{itemize}
\tightlist
\item
  Invite thoughts, critiques, and suggestions.
\end{itemize}

\subsection{Outlook --- Outlook \& Next Steps / Further
Research}\label{outlook-outlook-next-steps-further-research-1}

\textsubscript{Source:
\href{https://VJMeyer.github.io/submission/chapters/Conclusion.qmd.html\#f033ce82-1351-41e0-8370-a2b9558827ea}{Conclusion}}

\section*{Bibliography/References}\label{bibliographyreferences}
\addcontentsline{toc}{section}{Bibliography/References}

\phantomsection\label{refs}

\chapter*{Prefatory Apparatus: Illustrations and Terminology --- Quick
References}\label{prefatory-apparatus-illustrations-and-terminology-quick-references}
\addcontentsline{toc}{chapter}{Prefatory Apparatus: Illustrations and
Terminology --- Quick References}

\section*{List of Tables}\label{list-of-tables}
\addcontentsline{toc}{section}{List of Tables}

Table 1: Table name

Table 2: Table name

Table 3: Table name

\section*{List of Graphics \& Figures}\label{list-of-graphics-figures}
\addcontentsline{toc}{section}{List of Graphics \& Figures}

\section*{List of Abbreviations}\label{list-of-abbreviations}
\addcontentsline{toc}{section}{List of Abbreviations}

esp.~especially

f., ff.~following

incl.~including

p., pp.~page(s)

MAD Mutually Assured Destruction

\section*{Glossary}\label{glossary}
\addcontentsline{toc}{section}{Glossary}

\begin{description}
\item[term]
Definition of term
\item[Another term]
Description of second term
\end{description}

Text

\chapter{Appendices}\label{appendices}

\section{Appendices}\label{appendices-1}

\section{Appendix A}\label{appendix-a}

\section{Appendix B}\label{appendix-b}

\section{Appendix C}\label{appendix-c}

\section{Appendix D}\label{appendix-d}

TestText

\textsubscript{Source:
\href{https://VJMeyer.github.io/submission/chapters/Appendices.qmd.html\#66f55ce1-d4e4-4377-be30-005a37fa8c63}{Appendices}}

\chapter*{Notebooks}\label{notebooks}
\addcontentsline{toc}{chapter}{Notebooks}

% Add affidavit at the end but still within document environment
% \pagenumbering{Roman} % Switch to Roman page numbering

\clearpage
\thispagestyle{empty} % Removes page numbering for current page

\newpage


% Top header with logo (left) and department (right)
\begin{minipage}{0.3\textwidth}
  \includegraphics[width=5cm]{latex/uni-bayreuth-logo.png}
\end{minipage}
\hfill
\begin{minipage}{0.9\textwidth}
  \begin{center}
    -- P\&E Master's Programme --\\
    Chair of Philosophy, Computer\\
    Science \& Artificial Intelligence
  \end{center}
\end{minipage}

% Horizontal rule
\vspace{1.5cm}
\hrule
\vspace{2.5cm}

% Title in bold

  \LARGE\textbf{Affidavit}
\vspace{1.5cm}

\center

\normalsize

% \part*{Affidavit}

    \subsection*{\Large Declaration of Academic Honesty}
	    \vspace{1cm}\noindent \\
	    Hereby, I attest that I have composed and written the presented thesis 
        \vspace*{0.5cm}\noindent \\
        \textit{ \textbf{ Automating the Modelling of Transformative Artificial Intelligence Risks }}
        \vspace*{0.5cm}\noindent \\
        independently on my own, without the use of other than the stated aids and without any other resources than the ones indicated. All thoughts taken directly or indirectly from external sources are properly denoted as such.
	    \vspace{\baselineskip}
	    \\  This paper has neither been previously submitted in the same or a similar form to another authority nor has it been published yet.
	    \vspace{2cm}
	    
    \flushright
    \begin{minipage}{0.5\textwidth}
        \begin{flushleft} \large
        \textsc{Bayreuth}                     %   Place
        on the \\ % 26th of May 2025     \\
        \today           %   Date
        \vspace{2cm}\\
    	{\rule[-3pt]{\linewidth}{.4pt}\par\smallskip  
        \textsc{Valentin Meyer}	\\         %   Your name
    	}
        \end{flushleft}
        \end{minipage}


\end{document}


% % After loading packages but before \begin{document}
% \AtBeginDocument{
%   % Set up page styles
%   \fancypagestyle{frontmatter}{
%     \fancyhf{}
%     \fancyfoot[C]{\thepage}
%     \renewcommand{\footrulewidth}{0pt}
%   }

%   \fancypagestyle{mainmatter}{
%     \fancyhf{}
%     \fancyhead[LE,RO]{\slshape\nouppercase{\rightmark}}
%     \fancyhead[LO,RE]{\slshape\nouppercase{\leftmark}}
%     \fancyfoot[C]{\thepage}
%   }
% }

% % Inside the document, after title page
% \frontmatter
% \pagestyle{frontmatter}

% % Added automatically before Introduction
% % \mainmatter
% % \pagestyle{mainmatter}
