% Main template for University of Bayreuth thesis
\documentclass[12pt,a4paper]{report}      %   vs book

% Import custom packages and settings


% Set page geometry
\usepackage[a4paper,margin=2.5cm]{geometry}
\usepackage{graphicx}
\usepackage{booktabs}
\usepackage{bookmark}
\usepackage{hyperref}

% Define custom commands for metadata
\newcommand{\subtitle}{}
\newcommand{\supervisorname}{Supervisor Name}
\newcommand{\fieldofstudy}{}
\newcommand{\matriculationnumber}{}
\newcommand{\submissiondate}{}
\newcommand{\wordcount}{}
\providecommand{\tightlist}{%
  \setlength{\itemsep}{0pt}\setlength{\parskip}{0pt}}

% Standard document begins
\begin{document}

% Create custom title page
\begin{titlepage}
\thispagestyle{empty}% Remove page number from title page

% Top header with logo (left) and department (right)
\begin{minipage}{0.3\textwidth}
  \includegraphics[width=5cm]{latex/uni-bayreuth-logo.png}
\end{minipage}
\hfill
\begin{minipage}{0.9\textwidth}
  \begin{center}
    -- P\&E Master's Programme --\\
    Chair of Philosophy, Computer\\
    Science \& Artificial Intelligence
  \end{center}
\end{minipage}

% Horizontal rule
\vspace{1.5cm}
\hrule
\vspace{2cm}

% Title in bold
\begin{center}
  \Large\textbf{Automating the Modelling of
Transformative Artificial Intelligence Risks}
\end{center}
\vspace{0.2cm}

\begin{center}
  -----
\end{center}
\vspace{0.2cm}

% Subtitle in italics with quotation marks
\begin{center}
  \normalsize``\textit{An Epistemic Framework for Leveraging Frontier AI Systems
to Upscale Conditional Policy Assessments in Bayesian Networks on a Narrow Path towards Existencial Safety }''
\end{center}
\vspace{0.2cm}

\begin{center}
  -----
\end{center}
\vspace{0.2cm}

% Thesis designation
\begin{center}
  A thesis submitted at the Department of Philosophy\\[0.4cm]
  for the degree of \textit{Master of Arts in Philosophy \& Economics}
\end{center}

\vspace{1.5cm}
% Horizontal rule
\hrule
\vspace{1.5cm}

% Author and supervisor information with precise alignment
\begin{minipage}[t]{0.48\textwidth}
  \textbf{Author:}\\[0.3cm]
  Valentin Jakob Meyer\\
  Valentin.meyer@uni-bayreuth.de\\
  \textit{Matriculation Number:} 1828610\\
  \textit{Tel.:} +49 (1573) 4512494\\
  Pielmühler Straße 15\\
  52066 Lappersdorf
\end{minipage}
\hfill
\begin{minipage}[t]{0.48\textwidth}
  \begin{flushright}
    \textbf{Supervisor:}\\[0.3cm]
    Dr. Timo Speith\\[0.3cm]
    \textit{Word Count:}\\
    30.000\\[0.15cm]
    \textit{Source / Identifier:}\\
    Document URL
  \end{flushright}
\end{minipage}

% Date at bottom
\vfill
\begin{center}
  26th of May 2025
\end{center}
\end{titlepage}

% Rest of document
\bookmarksetup{startatroot}

\chapter*{Preface}\label{preface}
\addcontentsline{toc}{chapter}{Preface}

\markboth{Preface}{Preface}

This is a Quarto book.

To learn more about Quarto books visit
\url{https://quarto.org/docs/books}.

\bookmarksetup{startatroot}

\chapter*{Abstract}\label{abstract}
\addcontentsline{toc}{chapter}{Abstract}

\markboth{Abstract}{Abstract}

\bookmarksetup{startatroot}

\chapter*{Outline(s): Table of
Contents}\label{outlines-table-of-contents}
\addcontentsline{toc}{chapter}{Outline(s): Table of Contents}

\markboth{Outline(s): Table of Contents}{Outline(s): Table of Contents}

\bookmarksetup{startatroot}

\chapter*{Frontmatter}\label{frontmatter}
\addcontentsline{toc}{chapter}{Frontmatter}

\markboth{Frontmatter}{Frontmatter}

\bookmarksetup{startatroot}

\chapter*{Prefatory Apparatus: Illustrations and Terminology --- Quick
References}\label{prefatory-apparatus-illustrations-and-terminology-quick-references}
\addcontentsline{toc}{chapter}{Prefatory Apparatus: Illustrations and
Terminology --- Quick References}

\markboth{Prefatory Apparatus: Illustrations and Terminology --- Quick
References}{Prefatory Apparatus: Illustrations and Terminology --- Quick
References}

\section*{List of Tables}\label{list-of-tables}
\addcontentsline{toc}{section}{List of Tables}

\markright{List of Tables}

Table 1: Table name

Table 2: Table name

Table 3: Table name

\section*{List of Graphics \& Figures}\label{list-of-graphics-figures}
\addcontentsline{toc}{section}{List of Graphics \& Figures}

\markright{List of Graphics \& Figures}

\section*{List of Abbreviations}\label{list-of-abbreviations}
\addcontentsline{toc}{section}{List of Abbreviations}

\markright{List of Abbreviations}

esp.~especially

f., ff.~following

incl.~including

p., pp.~page(s)

MAD Mutually Assured Destruction

\section*{Glossary}\label{glossary}
\addcontentsline{toc}{section}{Glossary}

\markright{Glossary}

\bookmarksetup{startatroot}

\chapter{Introduction}\label{introduction}

\section{Introduction}\label{introduction-1}

10\% of Grade:

• introduces and motivates the core question or problem • provides
context for discussion (places issue within a larger debate or sphere of
relevance) • states precise thesis or position the author will argue for
• provides roadmap indicating structure and key content points of the
essay

\textasciitilde{} 14\% of text \textasciitilde{} 4200 words

• introduces and motivates the core question or problem

\section{Motivation: Problem
Statement}\label{motivation-problem-statement}

\section{Motivation: Research
Question}\label{motivation-research-question}

• provides context for discussion (places issue within a larger debate
or sphere of relevance)

\section{Scope: Aim \& Context of the
Research}\label{scope-aim-context-of-the-research}

\section{Significance of the Research: Theory of
Change}\label{significance-of-the-research-theory-of-change}

• states precise thesis or position the author will argue for

\section{Thesis Statement \& Position: (Aim of the
Paper)}\label{thesis-statement-position-aim-of-the-paper}

• provides roadmap indicating structure and key content points of the
essay

\section{Overview: Structure \& Approach of the Paper (Roadmap ---
Theory of
Change)}\label{overview-structure-approach-of-the-paper-roadmap-theory-of-change}

\section{Table of Contents}\label{table-of-contents}

\bookmarksetup{startatroot}

\chapter{Context}\label{context}

\section{Context}\label{context-1}

20\% of Grade:

• demonstrates understanding of all relevant core concepts • explains
why the question/thesis/problem is relevant in student's own words
(supported by quotations) • situates it within the debate/course
material • reconstructs selected arguments and identifies relevant
assumptions • describes additional relevant material that has been
consulted and integrates it with the course material as well as the
research question/thesis/problem

\textasciitilde{} 29\% of text \textasciitilde{} 8700 words

\begin{enumerate}
\def\labelenumi{\arabic{enumi}.}
\tightlist
\item
  successively (chunk my chunk) introduce concepts/ideas --- and 2.
  ground each with existing literature
\end{enumerate}

\bookmarksetup{startatroot}

\chapter{AMTAIR}\label{amtair}

\section{AMTAIR}\label{amtair-1}

20\% of Grade:

• provides critical or constructive evaluation of positions introduced •
develops strong (plausible) argument in support of author's own
position/thesis • argument draws on relevant course material •
claim/argument demonstrates understanding of the course materials incl.
key arguments and core concepts within the debate • claim/argument is
original or insightful, possibly even presents an original contribution
to the debate

\textasciitilde{} 29\% of text \textasciitilde{} 8700 words

\bookmarksetup{startatroot}

\chapter{Discussion}\label{discussion}

\section{Discussion}\label{discussion-1}

10\% of Grade:

• discusses a specific objection to student's own argument • provides a
convincing reply that bolsters or refines the main argument • relates to
or extends beyond materials/arguments covered in class

\textasciitilde{} 14\% of text \textasciitilde{} 4200 words

\bookmarksetup{startatroot}

\chapter{Conclusion}\label{conclusion}

\section{Conclusion}\label{conclusion-1}

10\% of Grade:

• summarizes thesis and line of argument • outlines possible
implications • notes outstanding issues / limitations of discussion •
points to avenues for further research • overall conclusion is in line
with introduction

\textasciitilde{} 14\% of text \textasciitilde{} 4200 words

\bookmarksetup{startatroot}

\chapter*{References}\label{references}
\addcontentsline{toc}{chapter}{References}

\markboth{References}{References}

\phantomsection\label{refs}

\cleardoublepage
\phantomsection
\addcontentsline{toc}{part}{Appendices}
\appendix

\chapter{Appendices}\label{appendices-1}

\section{Appendices}\label{appendices-2}

\section{Appendix A}\label{appendix-a}

\section{Appendix B}\label{appendix-b}

\section{Appendix C}\label{appendix-c}

\section{Appendix D}\label{appendix-d}

TestText

\chapter{appendixA}\label{appendixa}

testtext

% Add affidavit at the end but still within document environment

\clearpage
\thispagestyle{empty} % Removes page numbering for current page

\newpage


% Top header with logo (left) and department (right)
\begin{minipage}{0.3\textwidth}
  \includegraphics[width=5cm]{latex/uni-bayreuth-logo.png}
\end{minipage}
\hfill
\begin{minipage}{0.9\textwidth}
  \begin{center}
    -- P\&E Master's Programme --\\
    Chair of Philosophy, Computer\\
    Science \& Artificial Intelligence
  \end{center}
\end{minipage}

% Horizontal rule
\vspace{1.5cm}
\hrule
\vspace{2.5cm}

% Title in bold

  \LARGE\textbf{Affidavit}
\vspace{1.5cm}

\center

\normalsize

% \part*{Affidavit}

    \subsection*{\Large Declaration of Academic Honesty}
	    \vspace{1cm}\noindent \\
	    Hereby, I attest that I have composed and written the presented thesis 
        \vspace*{0.5cm}\noindent \\
        \textit{ \textbf{ Automating the Modelling of Transformative Artificial Intelligence Risks }}
        \vspace*{0.5cm}\noindent \\
        independently on my own, without the use of other than the stated aids and without any other resources than the ones indicated. All thoughts taken directly or indirectly from external sources are properly denoted as such.
	    \vspace{\baselineskip}
	    \\  This paper has neither been previously submitted in the same or a similar form to another authority nor has it been published yet.
	    \vspace{2cm}
	    
    \flushright
    \begin{minipage}{0.5\textwidth}
        \begin{flushleft} \large
        \textsc{Bayreuth}                     %   Place
        on the \\ % 26th of May 2025     \\
        \today           %   Date
        \vspace{2cm}\\
    	{\rule[-3pt]{\linewidth}{.4pt}\par\smallskip  
        \textsc{Valentin Meyer}	\\         %   Your name
    	}
        \end{flushleft}
        \end{minipage}


\end{document}


% % After loading packages but before \begin{document}
% \AtBeginDocument{
%   % Set up page styles
%   \fancypagestyle{frontmatter}{
%     \fancyhf{}
%     \fancyfoot[C]{\thepage}
%     \renewcommand{\footrulewidth}{0pt}
%   }

%   \fancypagestyle{mainmatter}{
%     \fancyhf{}
%     \fancyhead[LE,RO]{\slshape\nouppercase{\rightmark}}
%     \fancyhead[LO,RE]{\slshape\nouppercase{\leftmark}}
%     \fancyfoot[C]{\thepage}
%   }
% }

% % Inside the document, after title page
% \frontmatter
% \pagestyle{frontmatter}

% % Added automatically before Introduction
% % \mainmatter
% % \pagestyle{mainmatter}
