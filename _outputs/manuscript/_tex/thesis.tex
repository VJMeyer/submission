% Main template for University of Bayreuth thesis
\documentclass[12pt,a4paper]{report}      %   vs report

% Import custom packages and settings


% Set page geometry
\usepackage[a4paper,margin=2.5cm]{geometry}
\usepackage{graphicx}
\usepackage{booktabs}
\usepackage{bookmark}
\usepackage{hyperref}

% Define custom commands for metadata
\newcommand{\subtitle}{An Epistemic Framework for Leveraging Frontier AI
Systems to Upscale Conditional Policy Assessments in Bayesian Networks
on a Narrow Path towards Existential Safety}
\newcommand{\supervisorname}{Supervisor Name}
\newcommand{\fieldofstudy}{Philosophy \& Economics M.A.}
\newcommand{\matriculationnumber}{1828610}
\newcommand{\submissiondate}{May 26, 2025}
\newcommand{\wordcount}{30000}
\providecommand{\tightlist}{%
  \setlength{\itemsep}{0pt}\setlength{\parskip}{0pt}}

% Standard document begins
\begin{document}

% Create custom title page
\begin{titlepage}
\thispagestyle{empty}% Remove page number from title page

% Top header with logo (left) and department (right)
\begin{minipage}{0.3\textwidth}
  \includegraphics[width=5cm]{latex/uni-bayreuth-logo.png}
\end{minipage}
\hfill
\begin{minipage}{0.9\textwidth}
  \begin{center}
    -- P\&E Master's Programme --\\
    Chair of Philosophy, Computer\\
    Science \& Artificial Intelligence
  \end{center}
\end{minipage}

% Horizontal rule
\vspace{1.5cm}
\hrule
\vspace{2cm}

% Title in bold
\begin{center}
  \Large\textbf{Automating the Modelling of
Transformative Artificial Intelligence Risks}
\end{center}
\vspace{0.2cm}

\begin{center}
  -----
\end{center}
\vspace{0.2cm}

% Subtitle in italics with quotation marks
\begin{center}
  \normalsize``\textit{An Epistemic Framework for Leveraging Frontier AI Systems
to Upscale Conditional Policy Assessments in Bayesian Networks on a Narrow Path towards Existencial Safety }''
\end{center}
\vspace{0.2cm}

\begin{center}
  -----
\end{center}
\vspace{0.2cm}

% Thesis designation
\begin{center}
  A thesis submitted at the Department of Philosophy\\[0.4cm]
  for the degree of \textit{Master of Arts in Philosophy \& Economics}
\end{center}

\vspace{1.5cm}
% Horizontal rule
\hrule
\vspace{1.5cm}

% Author and supervisor information with precise alignment
\begin{minipage}[t]{0.48\textwidth}
  \textbf{Author:}\\[0.3cm]
  Valentin Jakob Meyer\\
  Valentin.meyer@uni-bayreuth.de\\
  \textit{Matriculation Number:} 1828610\\
  \textit{Tel.:} +49 (1573) 4512494\\
  Pielmühler Straße 15\\
  52066 Lappersdorf
\end{minipage}
\hfill
\begin{minipage}[t]{0.48\textwidth}
  \begin{flushright}
    \textbf{Supervisor:}\\[0.3cm]
    Dr. Timo Speith\\[0.3cm]
    \textit{Word Count:}\\
    30.000\\[0.15cm]
    \textit{Source / Identifier:}\\
    Document URL
  \end{flushright}
\end{minipage}

% Date at bottom
\vfill
\begin{center}
  26th of May 2025
\end{center}
\end{titlepage}

% Rest of document
\chapter*{Abstract}\label{abstract}
\addcontentsline{toc}{chapter}{Abstract}

The coordination crisis in AI governance presents a paradoxical
challenge: unprecedented investment in AI safety coexists alongside
fundamental coordination failures across technical, policy, and ethical
domains. These divisions systematically increase existential risk by
creating safety gaps, misallocating resources, and fostering
inconsistent approaches to interdependent problems. This thesis
introduces AMTAIR (Automating Transformative AI Risk Modeling), a
computational approach that addresses this coordination failure by
automating the extraction of probabilistic world models from AI safety
literature using frontier language models.

The AMTAIR system implements an end-to-end pipeline that transforms
unstructured text into interactive Bayesian networks through a novel
two-stage extraction process: first capturing argument structure in
ArgDown format, then enhancing it with probability information in
BayesDown. This approach bridges communication gaps between stakeholders
by making implicit models explicit, enabling comparison across different
worldviews, providing a common language for discussing probabilistic
relationships, and supporting policy evaluation across diverse
scenarios.

\textsubscript{Source:
\href{https://VJMeyer.github.io/submission/thesis.qmd.html}{Article
Notebook}}

\section{Callout Test --- Language \&
Style}\label{callout-test-language-style}

• employs appropriate tone and academic language • uses effective and
sophisticated sentence variety, diction, and vocabulary • employs
correct English spelling and grammar • is clearly written and uses
appropriate sentence complexity • communicates main points effectively /
is easy to follow • formats citations and references correctly and
consistently (e.g.~(AUTHOR, YEAR) citation style) • names all primary
and secondary sources • includes a complete list of references with full
bibliographic details

More text

\chapter{Introduction}\label{introduction}

\section{Introduction}\label{introduction-1}

10\% of Grade:

• introduces and motivates the core question or problem • provides
context for discussion (places issue within a larger debate or sphere of
relevance) • states precise thesis or position the author will argue for
• provides roadmap indicating structure and key content points of the
essay

\textasciitilde{} 14\% of text \textasciitilde{} 4200 words

• introduces and motivates the core question or problem

\section{Motivation: Problem
Statement}\label{motivation-problem-statement}

\section{Motivation: Research
Question}\label{motivation-research-question}

• provides context for discussion (places issue within a larger debate
or sphere of relevance)

\section{Scope: Aim \& Context of the
Research}\label{scope-aim-context-of-the-research}

\section{Significance of the Research: Theory of
Change}\label{significance-of-the-research-theory-of-change}

• states precise thesis or position the author will argue for

\section{Thesis Statement \& Position: (Aim of the
Paper)}\label{thesis-statement-position-aim-of-the-paper}

• provides roadmap indicating structure and key content points of the
essay

\section{Overview: Structure \& Approach of the Paper (Roadmap ---
Theory of
Change)}\label{overview-structure-approach-of-the-paper-roadmap-theory-of-change}

\section{Table of Contents}\label{table-of-contents}

\textsubscript{Source:
\href{https://VJMeyer.github.io/submission/chapters/Introduction.qmd.html\#1511baea-a705-44c1-94e9-c6ee72c34c3f}{Introduction}}

\chapter{Context}\label{context}

\section{Context}\label{context-1}

20\% of Grade:

• demonstrates understanding of all relevant core concepts • explains
why the question/thesis/problem is relevant in student's own words
(supported by quotations) • situates it within the debate/course
material • reconstructs selected arguments and identifies relevant
assumptions • describes additional relevant material that has been
consulted and integrates it with the course material as well as the
research question/thesis/problem

\textasciitilde{} 29\% of text \textasciitilde{} 8700 words

\begin{enumerate}
\def\labelenumi{\arabic{enumi}.}
\tightlist
\item
  successively (chunk my chunk) introduce concepts/ideas --- and 2.
  ground each with existing literature
\end{enumerate}

\textsubscript{Source:
\href{https://VJMeyer.github.io/submission/chapters/Context.qmd.html\#2e4caf53-0994-4a2c-beb1-296321648f04}{Context}}

\chapter{AMTAIR}\label{amtair}

\section{AMTAIR}\label{amtair-1}

20\% of Grade:

• provides critical or constructive evaluation of positions introduced •
develops strong (plausible) argument in support of author's own
position/thesis • argument draws on relevant course material •
claim/argument demonstrates understanding of the course materials incl.
key arguments and core concepts within the debate • claim/argument is
original or insightful, possibly even presents an original contribution
to the debate

\textasciitilde{} 29\% of text \textasciitilde{} 8700 words

\textsubscript{Source:
\href{https://VJMeyer.github.io/submission/chapters/AMTAIR.qmd.html\#050f6e6a-d523-40ce-ab7d-0063dfb6e0d0}{AMTAIR}}

\chapter{Discussion}\label{discussion}

\section{Discussion}\label{discussion-1}

10\% of Grade:

• discusses a specific objection to student's own argument • provides a
convincing reply that bolsters or refines the main argument • relates to
or extends beyond materials/arguments covered in class

\textasciitilde{} 14\% of text \textasciitilde{} 4200 words

\textsubscript{Source:
\href{https://VJMeyer.github.io/submission/chapters/Discussion.qmd.html\#6a205f09-6bca-40a9-a723-98d31e3907f4}{Discussion}}

\chapter{Conclusion}\label{conclusion}

\section{Conclusion}\label{conclusion-1}

10\% of Grade:

• summarizes thesis and line of argument • outlines possible
implications • notes outstanding issues / limitations of discussion •
points to avenues for further research • overall conclusion is in line
with introduction

\textasciitilde{} 14\% of text \textasciitilde{} 4200 words

\textsubscript{Source:
\href{https://VJMeyer.github.io/submission/chapters/Conclusion.qmd.html\#cf16ed96-aa1a-4d5f-a93d-d5f1b524cf26}{Conclusion}}

\section*{Bibliography/References}\label{bibliographyreferences}
\addcontentsline{toc}{section}{Bibliography/References}

\phantomsection\label{refs}

\chapter*{Prefatory Apparatus: Illustrations and Terminology --- Quick
References}\label{prefatory-apparatus-illustrations-and-terminology-quick-references}
\addcontentsline{toc}{chapter}{Prefatory Apparatus: Illustrations and
Terminology --- Quick References}

\section*{List of Tables}\label{list-of-tables}
\addcontentsline{toc}{section}{List of Tables}

Table 1: Table name

Table 2: Table name

Table 3: Table name

\section*{List of Graphics \& Figures}\label{list-of-graphics-figures}
\addcontentsline{toc}{section}{List of Graphics \& Figures}

\section*{List of Abbreviations}\label{list-of-abbreviations}
\addcontentsline{toc}{section}{List of Abbreviations}

esp.~especially

f., ff.~following

incl.~including

p., pp.~page(s)

MAD Mutually Assured Destruction

\section*{Glossary}\label{glossary}
\addcontentsline{toc}{section}{Glossary}

\begin{description}
\item[term]
Definition of term
\item[Another term]
Description of second term
\end{description}

Text

\chapter{Appendices}\label{appendices}

\section{Appendices}\label{appendices-1}

\section{Appendix A}\label{appendix-a}

\section{Appendix B}\label{appendix-b}

\section{Appendix C}\label{appendix-c}

\section{Appendix D}\label{appendix-d}

TestText

\section{Affidavit}\label{affidavit}

\textsubscript{Source:
\href{https://VJMeyer.github.io/submission/chapters/Appendices.qmd.html\#23997cc1-3a93-4cef-89c1-6b6ba523207a}{Appendices}}

\chapter*{Notebooks}\label{notebooks}
\addcontentsline{toc}{chapter}{Notebooks}

\end{document}


% % After loading packages but before \begin{document}
% \AtBeginDocument{
%   % Set up page styles
%   \fancypagestyle{frontmatter}{
%     \fancyhf{}
%     \fancyfoot[C]{\thepage}
%     \renewcommand{\footrulewidth}{0pt}
%   }

%   \fancypagestyle{mainmatter}{
%     \fancyhf{}
%     \fancyhead[LE,RO]{\slshape\nouppercase{\rightmark}}
%     \fancyhead[LO,RE]{\slshape\nouppercase{\leftmark}}
%     \fancyfoot[C]{\thepage}
%   }
% }

% % Inside the document, after title page
% \frontmatter
% \pagestyle{frontmatter}

% % Added automatically before Introduction
% % \mainmatter
% % \pagestyle{mainmatter}
