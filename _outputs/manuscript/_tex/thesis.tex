% Main template for University of Bayreuth thesis
\documentclass[12pt,a4paper]{book}      %   vs report

% Import custom packages and settings


% Set page geometry
\usepackage[a4paper,margin=2.5cm]{geometry}
\usepackage{graphicx}
\usepackage{booktabs}
\usepackage{bookmark}
\usepackage{hyperref}

% Define custom commands for metadata
\newcommand{\subtitle}{An Epistemic Framework for Leveraging Frontier AI
Systems to Upscale Conditional Policy Assessments in Bayesian Networks
on a Narrow Path towards Existential Safety}
\newcommand{\supervisorname}{Supervisor Name}
\newcommand{\fieldofstudy}{Philosophy \& Economics M.A.}
\newcommand{\matriculationnumber}{1828610}
\newcommand{\submissiondate}{May 26, 2025}
\newcommand{\wordcount}{30000}
\providecommand{\tightlist}{%
  \setlength{\itemsep}{0pt}\setlength{\parskip}{0pt}}

% Standard document begins
\begin{document}

% Create custom title page
\begin{titlepage}
\thispagestyle{empty}% Remove page number from title page

% Top header with logo (left) and department (right)
\begin{minipage}{0.3\textwidth}
  \includegraphics[width=5cm]{latex/uni-bayreuth-logo.png}
\end{minipage}
\hfill
\begin{minipage}{0.9\textwidth}
  \begin{center}
    -- P\&E Master's Programme --\\
    Chair of Philosophy, Computer\\
    Science \& Artificial Intelligence
  \end{center}
\end{minipage}

% Horizontal rule
\vspace{1.5cm}
\hrule
\vspace{2cm}

% Title in bold
\begin{center}
  \Large\textbf{Automating the Modelling of
Transformative Artificial Intelligence Risks}
\end{center}
\vspace{0.2cm}

\begin{center}
  -----
\end{center}
\vspace{0.2cm}

% Subtitle in italics with quotation marks
\begin{center}
  \normalsize``\textit{An Epistemic Framework for Leveraging Frontier AI Systems
to Upscale Conditional Policy Assessments in Bayesian Networks on a Narrow Path towards Existencial Safety }''
\end{center}
\vspace{0.2cm}

\begin{center}
  -----
\end{center}
\vspace{0.2cm}

% Thesis designation
\begin{center}
  A thesis submitted at the Department of Philosophy\\[0.4cm]
  for the degree of \textit{Master of Arts in Philosophy \& Economics}
\end{center}

\vspace{1.5cm}
% Horizontal rule
\hrule
\vspace{1.5cm}

% Author and supervisor information with precise alignment
\begin{minipage}[t]{0.48\textwidth}
  \textbf{Author:}\\[0.3cm]
  Valentin Jakob Meyer\\
  Valentin.meyer@uni-bayreuth.de\\
  \textit{Matriculation Number:} 1828610\\
  \textit{Tel.:} +49 (1573) 4512494\\
  Pielmühler Straße 15\\
  52066 Lappersdorf
\end{minipage}
\hfill
\begin{minipage}[t]{0.48\textwidth}
  \begin{flushright}
    \textbf{Supervisor:}\\[0.3cm]
    Dr. Timo Speith\\[0.3cm]
    \textit{Word Count:}\\
    30.000\\[0.15cm]
    \textit{Source / Identifier:}\\
    Document URL
  \end{flushright}
\end{minipage}

% Date at bottom
\vfill
\begin{center}
  26th of May 2025
\end{center}
\end{titlepage}

% Rest of document
\textsubscript{Source:
\href{https://VJMeyer.github.io/submission/thesis.qmd.html}{Article
Notebook}}

\chapter{}\label{section}

\chapter{Frontmatter}\label{frontmatter}

\chapter{Prefatory Apparatus: Illustrations and Terminology --- Quick
References}\label{prefatory-apparatus-illustrations-and-terminology-quick-references}

\section{List of Tables}\label{list-of-tables}

Table 1: Table name

Table 2: Table name

Table 3: Table name

\section{List of Graphics \& Figures}\label{list-of-graphics-figures}

\section{List of Abbreviations}\label{list-of-abbreviations}

esp.~especially

f., ff.~following

incl.~including

p., pp.~page(s)

MAD Mutually Assured Destruction

\section{Glossary}\label{glossary}

\textsubscript{Source:
\href{https://VJMeyer.github.io/submission/chapters/Frontmatter.qmd.html\#8315e7ab-034a-4a00-a97f-2592806b557d}{Frontmatter}}

\chapter{}\label{section-1}

\chapter{Introduction}\label{introduction}

10\% of Grade:

• introduces and motivates the core question or problem • provides
context for discussion (places issue within a larger debate or sphere of
relevance) • states precise thesis or position the author will argue for
• provides roadmap indicating structure and key content points of the
essay

\textasciitilde{} 14\% of text \textasciitilde{} 4200 words

• introduces and motivates the core question or problem

\section{Motivation: Problem
Statement}\label{motivation-problem-statement}

\section{Motivation: Research
Question}\label{motivation-research-question}

• provides context for discussion (places issue within a larger debate
or sphere of relevance)

\section{Scope: Aim \& Context of the
Research}\label{scope-aim-context-of-the-research}

\section{Significance of the Research: Theory of
Change}\label{significance-of-the-research-theory-of-change}

• states precise thesis or position the author will argue for

\section{Thesis Statement \& Position: (Aim of the
Paper)}\label{thesis-statement-position-aim-of-the-paper}

• provides roadmap indicating structure and key content points of the
essay

\section{Overview: Structure \& Approach of the Paper (Roadmap ---
Theory of
Change)}\label{overview-structure-approach-of-the-paper-roadmap-theory-of-change}

\section{Table of Contents}\label{table-of-contents}

\textsubscript{Source:
\href{https://VJMeyer.github.io/submission/chapters/Introduction.qmd.html\#23a3e2da-2445-4edd-96f4-8767e2857de6}{Introduction}}

\chapter{}\label{section-2}

\chapter{Context}\label{context}

20\% of Grade:

• demonstrates understanding of all relevant core concepts • explains
why the question/thesis/problem is relevant in student's own words
(supported by quotations) • situates it within the debate/course
material • reconstructs selected arguments and identifies relevant
assumptions • describes additional relevant material that has been
consulted and integrates it with the course material as well as the
research question/thesis/problem

\textasciitilde{} 29\% of text \textasciitilde{} 8700 words

\begin{enumerate}
\def\labelenumi{\arabic{enumi}.}
\tightlist
\item
  successively (chunk my chunk) introduce concepts/ideas --- and 2.
  ground each with existing literature
\end{enumerate}

\textsubscript{Source:
\href{https://VJMeyer.github.io/submission/chapters/Context.qmd.html\#41e533eb-06e4-4c34-b013-344a963258c1}{Context}}

\chapter{}\label{section-3}

\chapter{AMTAIR}\label{amtair}

20\% of Grade:

• provides critical or constructive evaluation of positions introduced •
develops strong (plausible) argument in support of author's own
position/thesis • argument draws on relevant course material •
claim/argument demonstrates understanding of the course materials incl.
key arguments and core concepts within the debate • claim/argument is
original or insightful, possibly even presents an original contribution
to the debate

\textasciitilde{} 29\% of text \textasciitilde{} 8700 words

\textsubscript{Source:
\href{https://VJMeyer.github.io/submission/chapters/AMTAIR.qmd.html\#66a99f1c-9dd6-40c4-97f6-f977c4a953f1}{AMTAIR}}

\chapter{}\label{section-4}

\chapter{Discussion}\label{discussion}

10\% of Grade:

• discusses a specific objection to student's own argument • provides a
convincing reply that bolsters or refines the main argument • relates to
or extends beyond materials/arguments covered in class

\textasciitilde{} 14\% of text \textasciitilde{} 4200 words

\textsubscript{Source:
\href{https://VJMeyer.github.io/submission/chapters/Discussion.qmd.html\#c5211e2d-0ed7-4ab0-838a-6545f2c41d91}{Discussion}}

\chapter{}\label{section-5}

\chapter{Conclusion}\label{conclusion}

10\% of Grade:

• summarizes thesis and line of argument • outlines possible
implications • notes outstanding issues / limitations of discussion •
points to avenues for further research • overall conclusion is in line
with introduction

\textasciitilde{} 14\% of text \textasciitilde{} 4200 words

\textsubscript{Source:
\href{https://VJMeyer.github.io/submission/chapters/Conclusion.qmd.html\#c14cf960-be7f-4a4a-a8f5-39f3c1bea077}{Conclusion}}

\section*{Bibliography/References}\label{bibliographyreferences}
\addcontentsline{toc}{section}{Bibliography/References}

\phantomsection\label{refs}

\chapter{}\label{section-6}

\chapter{Appendices}\label{appendices}

\chapter{Appendix A}\label{appendix-a}

\chapter{Appendix B}\label{appendix-b}

\chapter{Appendix C}\label{appendix-c}

\chapter{Appendix D}\label{appendix-d}

TestText

\chapter{Affidavit}\label{affidavit}

\textsubscript{Source:
\href{https://VJMeyer.github.io/submission/chapters/Appendices.qmd.html\#7b34ca3e-6a0a-4856-b99b-06c7c16c11c8}{Appendices}}

\chapter{Notebooks}\label{notebooks}

\end{document}
