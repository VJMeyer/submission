% Options for packages loaded elsewhere
% Options for packages loaded elsewhere
\PassOptionsToPackage{unicode}{hyperref}
\PassOptionsToPackage{hyphens}{url}
%
\documentclass[
  letterpaper,
]{book}
\usepackage{xcolor}
\usepackage{amsmath,amssymb}
\setcounter{secnumdepth}{5}
\usepackage{iftex}
\ifPDFTeX
  \usepackage[T1]{fontenc}
  \usepackage[utf8]{inputenc}
  \usepackage{textcomp} % provide euro and other symbols
\else % if luatex or xetex
  \usepackage{unicode-math} % this also loads fontspec
  \defaultfontfeatures{Scale=MatchLowercase}
  \defaultfontfeatures[\rmfamily]{Ligatures=TeX,Scale=1}
\fi
\usepackage{lmodern}
\ifPDFTeX\else
  % xetex/luatex font selection
\fi
% Use upquote if available, for straight quotes in verbatim environments
\IfFileExists{upquote.sty}{\usepackage{upquote}}{}
\IfFileExists{microtype.sty}{% use microtype if available
  \usepackage[]{microtype}
  \UseMicrotypeSet[protrusion]{basicmath} % disable protrusion for tt fonts
}{}
\makeatletter
\@ifundefined{KOMAClassName}{% if non-KOMA class
  \IfFileExists{parskip.sty}{%
    \usepackage{parskip}
  }{% else
    \setlength{\parindent}{0pt}
    \setlength{\parskip}{6pt plus 2pt minus 1pt}}
}{% if KOMA class
  \KOMAoptions{parskip=half}}
\makeatother
% Make \paragraph and \subparagraph free-standing
\makeatletter
\ifx\paragraph\undefined\else
  \let\oldparagraph\paragraph
  \renewcommand{\paragraph}{
    \@ifstar
      \xxxParagraphStar
      \xxxParagraphNoStar
  }
  \newcommand{\xxxParagraphStar}[1]{\oldparagraph*{#1}\mbox{}}
  \newcommand{\xxxParagraphNoStar}[1]{\oldparagraph{#1}\mbox{}}
\fi
\ifx\subparagraph\undefined\else
  \let\oldsubparagraph\subparagraph
  \renewcommand{\subparagraph}{
    \@ifstar
      \xxxSubParagraphStar
      \xxxSubParagraphNoStar
  }
  \newcommand{\xxxSubParagraphStar}[1]{\oldsubparagraph*{#1}\mbox{}}
  \newcommand{\xxxSubParagraphNoStar}[1]{\oldsubparagraph{#1}\mbox{}}
\fi
\makeatother


\usepackage{longtable,booktabs,array}
\usepackage{calc} % for calculating minipage widths
% Correct order of tables after \paragraph or \subparagraph
\usepackage{etoolbox}
\makeatletter
\patchcmd\longtable{\par}{\if@noskipsec\mbox{}\fi\par}{}{}
\makeatother
% Allow footnotes in longtable head/foot
\IfFileExists{footnotehyper.sty}{\usepackage{footnotehyper}}{\usepackage{footnote}}
\makesavenoteenv{longtable}
\usepackage{graphicx}
\makeatletter
\newsavebox\pandoc@box
\newcommand*\pandocbounded[1]{% scales image to fit in text height/width
  \sbox\pandoc@box{#1}%
  \Gscale@div\@tempa{\textheight}{\dimexpr\ht\pandoc@box+\dp\pandoc@box\relax}%
  \Gscale@div\@tempb{\linewidth}{\wd\pandoc@box}%
  \ifdim\@tempb\p@<\@tempa\p@\let\@tempa\@tempb\fi% select the smaller of both
  \ifdim\@tempa\p@<\p@\scalebox{\@tempa}{\usebox\pandoc@box}%
  \else\usebox{\pandoc@box}%
  \fi%
}
% Set default figure placement to htbp
\def\fps@figure{htbp}
\makeatother





\setlength{\emergencystretch}{3em} % prevent overfull lines

\providecommand{\tightlist}{%
  \setlength{\itemsep}{0pt}\setlength{\parskip}{0pt}}



 


% Required packages
\usepackage{graphicx}
\usepackage{geometry}
\usepackage{fancyhdr}
\usepackage{titling}
\usepackage{array}
\usepackage{booktabs}
\usepackage{xcolor}
\usepackage{hyperref}
\usepackage{setspace}
\usepackage{microtype}

% Better typography
\usepackage[protrusion=true,expansion=true]{microtype}

% Spacing
\onehalfspacing

% Header and footer styling
\pagestyle{fancy}
\fancyhf{}
\fancyhead[LE,RO]{\slshape\nouppercase{\rightmark}}
\fancyhead[LO,RE]{\slshape\nouppercase{\leftmark}}
\fancyfoot[C]{\thepage}
\makeatletter
\@ifpackageloaded{bookmark}{}{\usepackage{bookmark}}
\makeatother
\makeatletter
\@ifpackageloaded{caption}{}{\usepackage{caption}}
\AtBeginDocument{%
\ifdefined\contentsname
  \renewcommand*\contentsname{Table of contents}
\else
  \newcommand\contentsname{Table of contents}
\fi
\ifdefined\listfigurename
  \renewcommand*\listfigurename{List of Figures}
\else
  \newcommand\listfigurename{List of Figures}
\fi
\ifdefined\listtablename
  \renewcommand*\listtablename{List of Tables}
\else
  \newcommand\listtablename{List of Tables}
\fi
\ifdefined\figurename
  \renewcommand*\figurename{Figure}
\else
  \newcommand\figurename{Figure}
\fi
\ifdefined\tablename
  \renewcommand*\tablename{Table}
\else
  \newcommand\tablename{Table}
\fi
}
\@ifpackageloaded{float}{}{\usepackage{float}}
\floatstyle{ruled}
\@ifundefined{c@chapter}{\newfloat{codelisting}{h}{lop}}{\newfloat{codelisting}{h}{lop}[chapter]}
\floatname{codelisting}{Listing}
\newcommand*\listoflistings{\listof{codelisting}{List of Listings}}
\makeatother
\makeatletter
\makeatother
\makeatletter
\@ifpackageloaded{caption}{}{\usepackage{caption}}
\@ifpackageloaded{subcaption}{}{\usepackage{subcaption}}
\makeatother
\usepackage{bookmark}
\IfFileExists{xurl.sty}{\usepackage{xurl}}{} % add URL line breaks if available
\urlstyle{same}
\hypersetup{
  pdftitle={Automating the Modelling of Transformative Artificial Intelligence Risks},
  pdfauthor={Valentin Jakob Meyer},
  hidelinks,
  pdfcreator={LaTeX via pandoc}}


\title{Automating the Modelling of Transformative Artificial
Intelligence Risks}
\author{Valentin Jakob Meyer}
\date{2025-05-26}
\begin{document}
\frontmatter
\maketitle

\renewcommand*\contentsname{Table of contents}
{
\setcounter{tocdepth}{2}
\tableofcontents
}

\mainmatter
\bookmarksetup{startatroot}

\chapter*{Preface}\label{preface}
\addcontentsline{toc}{chapter}{Preface}

\markboth{Preface}{Preface}

This is a Quarto book.

To learn more about Quarto books visit
\url{https://quarto.org/docs/books}.

\bookmarksetup{startatroot}

\chapter*{Abstract}\label{abstract}
\addcontentsline{toc}{chapter}{Abstract}

\markboth{Abstract}{Abstract}

\bookmarksetup{startatroot}

\chapter*{Outline(s)}\label{outlines}
\addcontentsline{toc}{chapter}{Outline(s)}

\markboth{Outline(s)}{Outline(s)}

\bookmarksetup{startatroot}

\chapter*{Frontmatter}\label{frontmatter}
\addcontentsline{toc}{chapter}{Frontmatter}

\markboth{Frontmatter}{Frontmatter}

\bookmarksetup{startatroot}

\chapter*{Prefatory Apparatus: Illustrations and Terminology --- Quick
References}\label{prefatory-apparatus-illustrations-and-terminology-quick-references}
\addcontentsline{toc}{chapter}{Prefatory Apparatus: Illustrations and
Terminology --- Quick References}

\markboth{Prefatory Apparatus: Illustrations and Terminology --- Quick
References}{Prefatory Apparatus: Illustrations and Terminology --- Quick
References}

\section*{List of Tables}\label{list-of-tables}
\addcontentsline{toc}{section}{List of Tables}

\markright{List of Tables}

Table 1: Table name

Table 2: Table name

Table 3: Table name

\section*{List of Graphics \& Figures}\label{list-of-graphics-figures}
\addcontentsline{toc}{section}{List of Graphics \& Figures}

\markright{List of Graphics \& Figures}

\section*{List of Abbreviations}\label{list-of-abbreviations}
\addcontentsline{toc}{section}{List of Abbreviations}

\markright{List of Abbreviations}

esp.~especially

f., ff.~following

incl.~including

p., pp.~page(s)

MAD Mutually Assured Destruction

\section*{Glossary}\label{glossary}
\addcontentsline{toc}{section}{Glossary}

\markright{Glossary}

\bookmarksetup{startatroot}

\chapter{Introduction}\label{introduction}

10\% of Grade:

• introduces and motivates the core question or problem • provides
context for discussion (places issue within a larger debate or sphere of
relevance) • states precise thesis or position the author will argue for
• provides roadmap indicating structure and key content points of the
essay

\textasciitilde{} 14\% of text \textasciitilde{} 4200 words

• introduces and motivates the core question or problem

\section{Motivation: Problem
Statement}\label{motivation-problem-statement}

\section{Motivation: Research
Question}\label{motivation-research-question}

• provides context for discussion (places issue within a larger debate
or sphere of relevance)

\section{Scope: Aim \& Context of the
Research}\label{scope-aim-context-of-the-research}

\section{Significance of the Research: Theory of
Change}\label{significance-of-the-research-theory-of-change}

• states precise thesis or position the author will argue for

\section{Thesis Statement \& Position: (Aim of the
Paper)}\label{thesis-statement-position-aim-of-the-paper}

• provides roadmap indicating structure and key content points of the
essay

\section{Overview: Structure \& Approach of the Paper (Roadmap ---
Theory of
Change)}\label{overview-structure-approach-of-the-paper-roadmap-theory-of-change}

\section{Table of Contents}\label{table-of-contents}

\bookmarksetup{startatroot}

\chapter{Context}\label{context}

20\% of Grade:

• demonstrates understanding of all relevant core concepts • explains
why the question/thesis/problem is relevant in student's own words
(supported by quotations) • situates it within the debate/course
material • reconstructs selected arguments and identifies relevant
assumptions • describes additional relevant material that has been
consulted and integrates it with the course material as well as the
research question/thesis/problem

\textasciitilde{} 29\% of text \textasciitilde{} 8700 words

\begin{enumerate}
\def\labelenumi{\arabic{enumi}.}
\tightlist
\item
  successively (chunk my chunk) introduce concepts/ideas --- and 2.
  ground each with existing literature
\end{enumerate}

\bookmarksetup{startatroot}

\chapter{AMTAIR}\label{amtair}

20\% of Grade:

• provides critical or constructive evaluation of positions introduced •
develops strong (plausible) argument in support of author's own
position/thesis • argument draws on relevant course material •
claim/argument demonstrates understanding of the course materials incl.
key arguments and core concepts within the debate • claim/argument is
original or insightful, possibly even presents an original contribution
to the debate

\textasciitilde{} 29\% of text \textasciitilde{} 8700 words

\bookmarksetup{startatroot}

\chapter{Discussion}\label{discussion}

10\% of Grade:

• discusses a specific objection to student's own argument • provides a
convincing reply that bolsters or refines the main argument • relates to
or extends beyond materials/arguments covered in class

\textasciitilde{} 14\% of text \textasciitilde{} 4200 words

\bookmarksetup{startatroot}

\chapter{Conclusion}\label{conclusion}

10\% of Grade:

• summarizes thesis and line of argument • outlines possible
implications • notes outstanding issues / limitations of discussion •
points to avenues for further research • overall conclusion is in line
with introduction

\textasciitilde{} 14\% of text \textasciitilde{} 4200 words

\bookmarksetup{startatroot}

\chapter*{References}\label{references}
\addcontentsline{toc}{chapter}{References}

\markboth{References}{References}

\phantomsection\label{refs}

\cleardoublepage
\phantomsection
\addcontentsline{toc}{part}{Appendices}
\appendix

\chapter{Appendices}\label{appendices-1}

\chapter{Appendix A}\label{appendix-a}

\chapter{Appendix B}\label{appendix-b}

\chapter{Appendix C}\label{appendix-c}

\chapter{Appendix D}\label{appendix-d}

TestText

\chapter{Affidavit}\label{affidavit}

\chapter{}\label{section}

testtext


\backmatter


\end{document}
